\documentclass{beamer}



\usetheme{Madrid}
\usecolortheme{default}

\logo{{\includegraphics[width=0.1\textwidth]{../images/the-tech-group-pam-high-resolution-logo-black-on-transparent-background_01.png}}} 
\institute{Jkuat} 
\author{Martin Maina}
\date{\today}

\begin{document}


\title{Generating a Table of Contents in LaTeX}

\frame{\titlepage}

\begin{frame}
\frametitle{Table of Contents}
\tableofcontents
\end{frame}

\section{Generating a Table of Contents}

\begin{frame}
\frametitle{Generating a Table of Contents}
LaTeX offers features to automatically generate a table of contents, a list of figures, and a list of tables. It uses the section headings to create the table of contents.
\end{frame}

\section{List of Figures / Tables}

\begin{frame}
\frametitle{List of Figures / Tables}
The generation of a list of figures and tables works the same way. You can use the \texttt{\textbackslash listoffigures} and \texttt{\textbackslash listoftables} commands to create these lists.
\end{frame}

\section{Depth}

\begin{frame}
\frametitle{Depth}
Sometimes it's useful to show only a subset of headings in the table of contents. The \texttt{\textbackslash setcounter\{tocdepth\}\{X\}} command allows you to set the desired depth, where X is the level of depth you want to display.
\end{frame}

\section{Spacing}

\begin{frame}
\frametitle{Spacing}
If you want to change the spacing of the table of contents or the entire document, you can use the \texttt{setspace} package. It provides commands like \texttt{\textbackslash singlespacing} and \texttt{\textbackslash doublespacing} to adjust the spacing.
\end{frame}

\end{document}
