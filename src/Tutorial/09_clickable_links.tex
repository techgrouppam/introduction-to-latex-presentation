\documentclass{beamer}
\usepackage{hyperref}


\include{{beamer_header.tex}}

\title{How to Make Clickable Links in \LaTeX}


\begin{document}

\begin{frame}
  \titlepage
\end{frame}

\begin{frame}
\frametitle{Setting Hyperlinks}
Adding clickable links to \LaTeX{} documents is easy with the \texttt{hyperref} package. You only need to include it in your preamble:
\vspace{1em}
\texttt{\textbackslash usepackage\{hyperref\}}
\vspace{1em}
After including the package, you can add links to your document using the \texttt{\textbackslash href} command.
\end{frame}

\begin{frame}[fragile]
\frametitle{Adding Links}
To add a link with a description (making a word clickable), use the \texttt{\textbackslash href} command:
\vspace{1em}
\begin{verbatim}
This is my link: \href{http://www.github.com/techgrouppam}{GitHub}.
\end{verbatim}
\vspace{1em}
This will create a clickable link in your document.
\end{frame}

\begin{frame}[fragile]
\frametitle{Embedding Bare URLs}
If you want to embed a bare URL without an additional description, use the \texttt{\textbackslash url} command:
\vspace{1em}
\begin{verbatim}
You can also link to bare URLs: \url{http://www.github.com/techgrouppam}
\end{verbatim}
\vspace{1em}
This will display the URL as a clickable link.
\end{frame}

\begin{frame}[fragile]
\frametitle{Adding Email Addresses}
To add your email address as a clickable link, use the \texttt{\textbackslash href} command with the \texttt{mailto} protocol:
\vspace{1em}
\begin{verbatim}
My email address is:
\href{mailto:techgrouppam@gmail.com}{techgrouppam@gmail.com}
\end{verbatim}
\vspace{1em}
Clicking on the email address will open the reader's email program.
\end{frame}

\end{document}
