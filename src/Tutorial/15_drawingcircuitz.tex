\documentclass{beamer}

\usepackage{tikz}
\usepackage{circuitikz}

\include{{beamer_header.tex}}

\title{Circuit Diagrams in \LaTeX{} - Using Circuitikz}


\begin{document}

\begin{frame}
  \titlepage
\end{frame}

\begin{frame}{Introduction}
  Add neat circuit diagrams to your paper with Circuitikz, extending Tikz with electric components.
\end{frame}

\begin{frame}{Circuitikz Example}
  \begin{figure}[h!]
    \begin{center}
      \begin{circuitikz}
        \draw (0,0)
        to[V,v=$U_q$] (0,2)
        to[short] (2,2)
        to[R=$R_1$] (2,0)
        to[short] (0,0);
      \end{circuitikz}
      \caption{My first circuit.}
    \end{center}
  \end{figure}
\end{frame}

\begin{frame}{Drawing Circuits with Circuitikz}
  \begin{itemize}
    \item The circuit is drawn using the same syntax as Tikz.
    \item Special options are specified for the circuit elements.
    \item For example, to draw a voltage source, use \texttt{to[V,v=$U_q$]}.
    \item To draw a resistor, use \texttt{to[R=$R_1$]}.
  \end{itemize}
\end{frame}

\begin{frame}{Adding More Elements}
  \begin{figure}[h!]
    \begin{center}
      \begin{circuitikz}
        \draw (0,0)
        to[V,v=$U_q$] (0,2)
        to[short] (2,2)
        to[R=$R_1$] (2,0)
        to[short] (0,0);
        \draw (2,2)
        to[short] (4,2)
        to[L=$L_1$] (4,0)
        to[short] (2,0);
      \end{circuitikz}
      \caption{Circuit with inductor.}
    \end{center}
  \end{figure}
\end{frame}

\begin{frame}{Adding More Elements (contd.)}
  \begin{figure}[h!]
    \begin{center}
      \begin{circuitikz}
        \draw (0,0)
        to[V,v=$U_q$] (0,2)
        to[short] (2,2)
        to[R=$R_1$] (2,0)
        to[short] (0,0);
        \draw (2,2)
        to[short] (4,2)
        to[L=$L_1$] (4,0)
        to[short] (2,0);
        \draw (4,2)
        to[short] (6,2)
        to[C=$C_1$] (6,0)
        to[short] (4,0);
      \end{circuitikz}
      \caption{Circuit with inductor and capacitor.}
    \end{center}
  \end{figure}
\end{frame}

\begin{frame}{Conclusion}
  The Circuitikz package allows you to create circuit
\end{frame}
\end{document}