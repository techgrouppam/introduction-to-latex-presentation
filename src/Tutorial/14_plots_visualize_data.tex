\documentclass{beamer}
\usepackage{siunitx}
\include{{beamer_header.tex}}
\title{Plots in \LaTeX}
\usepackage{tikz}
\usepackage{pgfplots}
\pgfplotsset{compat=newest}
\usepgfplotslibrary{units}
\sisetup{
  round-mode          = places,
  round-precision     = 2,
}

\begin{document}
\begin{frame}
    \titlepage
\end{frame}
\begin{frame}
\frametitle{Plots in \LaTeX{} -- Visualize data with pgfplots}
\framesubtitle{Basic plotting}

\begin{figure}[h!]
  \begin{center}
    \begin{tikzpicture}[scale=0.5]
      \begin{axis}[
          width=\linewidth,
          grid=major,
          grid style={dashed,gray!30},
          xlabel=X Axis $U$,
          ylabel=Y Axis $I$,
          x unit=\si{\volt},
          y unit=\si{\ampere},
          legend style={at={(0.5,-0.2)},anchor=north},
          x tick label style={rotate=90,anchor=east}
        ]
        \addplot
        table[x=column 1,y=column 2,col sep=comma] {table.csv};
        \legend{Plot}
      \end{axis}
    \end{tikzpicture}
    \caption{My first autogenerated plot.}
  \end{center}
\end{figure}

\end{frame}

\begin{frame}
\frametitle{Packages and setup}

We added the new packages and the following code to generate the plot:

\begin{figure}[h!]
  \begin{center}
    \begin{tikzpicture}[scale=0.75]
      \begin{axis}[
          width=\linewidth,
          grid=major,
          grid style={dashed,gray!30},
          xlabel=X Axis $U$,
          ylabel=Y Axis $I$,
          x unit=\si{\volt},
          y unit=\si{\ampere},
          legend style={at={(0.5,-0.2)},anchor=north},
          x tick label style={rotate=90,anchor=east}
        ]
        \addplot
        table[x=column 1,y=column 2,col sep=comma] {table.csv};
        \legend{Plot}
      \end{axis}
    \end{tikzpicture}
    \caption{My first autogenerated plot.}
  \end{center}
\end{figure}

\end{frame}

\begin{frame}
\frametitle{The most important part}

Given a .csv file like:

\begin{table}
\centering
\begin{tabular}{cc}
column 1 & column 2 \\
1 & 2 \\
11.432 & 2342.23123123 \\
\end{tabular}
\end{table}

We have to put the name of one column for our x, in this case $x=\text{{column 1}}$, and a second column for our y, since there are only two columns, we choose $y=\text{{column 2}}$. Again, the \texttt{col sep=comma} indicates that we use a comma as our column separator.

\end{frame}

\end{document}
